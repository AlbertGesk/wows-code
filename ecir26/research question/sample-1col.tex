%% The first command in your LaTeX source must be the \documentclass command.
%%
%% Options:
%% twocolumn : Two column layout. Do not use twocolumn for papers submitted to CEUR-WS!
%% hf: enable header and footer.
\documentclass[
%twocolumn,
% hf,
]{ceurart}

%%
%% One can fix some overfulls
\sloppy

%%
%% Minted listings support 
%% Need pygment <http://pygments.org/> <http://pypi.python.org/pypi/Pygments>
\usepackage{listings}
%% auto break lines
\lstset{breaklines=true}

%%
%% end of the preamble, start of the body of the document source.
\begin{document}

%%
%% Rights management information.
%% CC-BY is default license.
\copyrightyear{2026}
\copyrightclause{Copyright for this paper by its authors.
  Use permitted under Creative Commons License Attribution 4.0
  International (CC BY 4.0).}

%%
%% This command is for the conference information
\conference{IR Lab Course, Winter Semester 2025}

%%
%% The "title" command
\title{A Comparative Study of Bo1 and RM3 Query Expansion using DPH Retrieval}

%%
%% The "author" command and its associated commands are used to define
%% the authors and their affiliations.
\author[1]{Albert Gesk}[%
email=albert.gesk@gmail.com,
url=https://github.com/AlbertGesk,
]
\author[1]{Lilian Erler}[%
email=uk076601@student.uni-kassel.de,
url=https://github.com/lilianerler,
]
\address[1]{University Kassel, Germany}


%% Footnotes
\cortext[1]{Corresponding author.}
\fntext[1]{These authors contributed equally.}

%%
%% The abstract is a short summary of the work to be presented in the
%% article.
\begin{abstract}
  Query expansion via pseudo-relevance feedback is a well-established technique to improve 
  retrieval effectiveness in information retrieval systems. Among classical approaches, 
  Bo1 and RM3 are frequently used but exhibit varying effectiveness depending on the 
  retrieval setting. In this work, we empirically compare Bo1 and RM3 query expansion 
  within a controlled Retriever-Rewriter-Retriever pipeline using the DPH retrieval model. 
  Experiments are conducted on the \textit{radboud-validation-20251114-training} dataset 
  using nDCG@10 as evaluation metric. Statistical significance testing shows that neither 
  a significant difference nor a significant improvement in retrieval effectiveness can be 
  observed between Bo1 and RM3 under identical parameter settings.
\end{abstract}

%%
%% Keywords. The author(s) should pick words that accurately describe
%% the work being presented. Separate the keywords with commas.
\begin{keywords}
  Information Retrieval \sep
  Query Expansion \sep
  Bo1 \sep
  RM3 \sep
  PyTerrier
\end{keywords}

%%
%% This command processes the author and affiliation and title
%% information and builds the first part of the formatted document.
\maketitle

\section{Introduction}

Query expansion using pseudo-relevance feedback (PRF) is a classical approach to mitigate 
vocabulary mismatch in information retrieval (IR) systems. The mechanism of pseudo-relevance 
feedback (PRF) involves extracting expansion terms from the initial top-ranked documents to 
formulate a new query for a second retrieval stage. Among commonly used PRF techniques, Bo1 and RM3
are widely adopted in traditional IR systems and toolkits.

Despite their popularity, the relative effectiveness of Bo1 and RM3 is known to be sensitive 
to retrieval models, parameter settings, and datasets. This motivates a controlled comparison 
of both methods under identical experimental conditions.

In this paper, we investigate the following research question: \textit{Does the choice of Bo1 versus RM3 query expansion lead to significant differences 
or improvements in retrieval effectiveness when applied within a DPH-based retrieval pipeline?}

Our contributions are:
(i) a controlled experimental comparison of Bo1 and RM3 query expansion using identical 
retrieval and feedback settings, and
(ii) a statistical analysis of their impact on nDCG@10 on the 
\textit{radboud-validation-20251114-training} dataset.


\section{Related Work}

\section{Methodology}
This section depicts the experimental setup designed to answer the research question.

\subsection{Dataset and Preprocessing}
For the experiments, we utilized the \textit{radboud-validation-20251114-training} dataset. 
The indexing of the documents was based on the standard textual representation (\texttt{default\_text}) 
provided in the dataset. No additional preprocessing steps or filtering were applied, 
meaning only the standard tokenization performed by PyTerrier was used. 

\subsection{Experimental Design}
\label{subsec:exp}
The experiment use a retriever-rewriter-retriever pipeline as foundation. The DPH retrieval model \cite{DBLP:conf/ecir/Amati06} implemented via PyTerrier 
is setup with a thousand maximum number of results to return per query and with no metadata fields to return for each search result. 
The pseudo-relevance feedback models,both the Bo1 \cite{DBLP:phd/ethos/Amati03} and RM3 \cite{DBLP:conf/trec/JaleelACDLLSW04} are implemented via PyTerrier as well.
For both models, all parameters are remained at default, the number of feedback terms to use is set to ten and the number of feedback documents to use is set to three. 
The interpolation weight between the original query and the feedback model for RM3 is set to 0.6.

\noindent To evaluate the effectiveness of the retrieval pipelines, differences in per-topic 
nDCG@10 scores were assessed for statistical significance using a paired Student’s 
t-test. The significance level was set to $\alpha = 0.05$, and Bonferroni correction 
was applied to account for multiple comparisons.

\subsection{Hypotheses and Null-Hypotheses}
Given the research question the following two hypotheses and corresponding null-hypotheses are formulated:
\begin{itemize}
  \item $H^{}_{1}$ Given query expansion via pseudo-relevance feedback using the top 3 retrieved documents and 10 expansion terms, there is a statistically significant difference in nDCG@10 on radboud-validation-20251114-training between (i) the DPH-based retrieval pipeline described in subsection~\ref{subsec:exp} three with Bo1 query expansion and (ii) the same pipeline with RM3 query expansion. ($\alpha$=0.05)
  \item $H^{0}_{1}$ Given query expansion via pseudo-relevance feedback using the top 3 retrieved documents and 10 expansion terms, there is no statistically significant difference in nDCG@10 on radboud-validation-20251114-training between (i) the DPH-based retrieval pipeline described in subsection~\ref{subsec:exp} three with Bo1 query expansion and (ii) the same pipeline with RM3 query expansion. ($\alpha$=0.05) 
  \item $H^{}_{2}$ Given query expansion via pseudo-relevance feedback using the top 3 retrieved documents and 10 expansion terms, there is a statistically significant improvement in nDCG@10 on radboud-validation-20251114-training between (i) the DPH-based retrieval pipeline described in subsection~\ref{subsec:exp} three with Bo1 query expansion and (ii) the same pipeline with RM3 query expansion. ($\alpha$=0.05)
  \item $H^{0}_{2}$ Given query expansion via pseudo-relevance feedback using the top 3 retrieved documents and 10 expansion terms, there is no statistically significant improvement in nDCG@10 on radboud-validation-20251114-training between (i) the DPH-based retrieval pipeline described in subsection~\ref{subsec:exp} three with Bo1 query expansion and (ii) the same pipeline with RM3 query expansion. ($\alpha$=0.05)
\end{itemize}


\section{Results}
Table~\ref{tab:results} presents the retrieval effectiveness and statistical testing results.
The DPH-Bo1-DPH pipeline achieves a higher mean nDCG@10 score than DPH-RM3-DPH. Though, Bo1 performs better than RM3. The p-value of 0.283 is significantly larger than 0.05. 
This means the observed difference is likely due to random variation across topics. 
We failed to reject our $H^{0}_{1}$ and $H^{0}_{2}$, which means we can't prove our $H^{}_{1}$ and $H^{}_{2}$.

\begin{table}[t]
\centering
\caption{Retrieval effectiveness and statistical significance (nDCG@10).}
\label{tab:results}
\begin{tabular}{lccc}
\toprule
Method & nDCG@10 & p-value \\
\midrule
DPH-Bo1-DPH & 0.4947 & \multirow{2}{*}{0.283} \\
DPH-RM3-DPH & 0.4744 &   \\
\bottomrule
\end{tabular}
\end{table}


\section{Conclusion}
Given the parameters of 3 retrieved documents and 10 expansion terms for Bo1 and RM3, 
both pseudo-relevance feedback-based transformer models behave similarly. 
Experimenting with more feedback documents and more expansion terms in the future could be meaningful.

\section{Reference}



















































\section{Template parameters}

There are a number of template
parameters which modify some part of the \verb|ceurart| document class.
This parameters are enclosed in square
brackets and are a part of the \verb|\documentclass| command:
\begin{lstlisting}
  \documentclass[parameter]{ceurart}
\end{lstlisting}

Frequently-used parameters, or combinations of parameters, include:
\begin{itemize}
\item \verb|twocolumn| : Two column layout. This option is not supported bey CEUR-WS, hence only use it for papers not to be submitted to CEUR-WS.
\item \verb|hf| : Enable header and footer\footnote{You can enable
    the display of page numbers in the final version of the entire
    collection. In this case, you should adhere to the end-to-end
    pagination of individual papers.}.  This option shall also not be used for papers to be published by CEUR-WS.
\end{itemize}

\section{Front matter}

\subsection{Title Information}

The titles of papers should be either all use the emphasizing
capitalized style or they should all use the regular English (or
native language) style. It does not make a good impression if you or
your authors mix the styles.

Use the \verb|\title| command to define the title of your work. Do not
insert line breaks in your title.

\subsection{Title variants}

\verb|\title| command have the below options:
\begin{itemize}
\item \verb|title|: Document title. This is default option. 
\begin{lstlisting}
\title[mode=title]{This is a title}
\end{lstlisting}
You can just omit it, like as follows:
\begin{lstlisting}
\title{This is a title}
\end{lstlisting}

\item \verb|alt|: Alternate title.
\begin{lstlisting}
\title[mode=alt]{This is a alternate title}
\end{lstlisting}

\item \verb|sub|: Sub title.
\begin{lstlisting}
\title[mode=sub]{This is a sub title}
\end{lstlisting}
You can just use \verb|\subtitle| command, as follows:
\begin{lstlisting}
\subtitle{This is a sub title}
\end{lstlisting}

\item \verb|trans|: Translated title.
\begin{lstlisting}
\title[mode=trans]{This is a translated title}
\end{lstlisting}

\item \verb|transsub|: Translated sub title.
\begin{lstlisting}
\title[mode=transsub]{This is a translated sub title}
\end{lstlisting}
\end{itemize}

\subsection{Authors and Affiliations}

Each author must be defined separately for accurate metadata
identification. Multiple authors may share one affiliation. Authors'
names should not be abbreviated; use full first names wherever
possible. Include authors' e-mail addresses whenever possible.

\verb|\author| command have the below options: 

\begin{itemize}
\item \verb|style| : Style of author name (chinese)
\item \verb|prefix| : Prefix
\item \verb|suffix| : Suffix
\item \verb|degree| : Degree
\item \verb|role| : Role
\item \verb|orcid| : ORCID
\item \verb|email| : E-mail
\item \verb|url| : URL
\end{itemize}

Author names can have some kinds of marks and notes:
\begin{itemize}
\item affiliation mark: \verb|\author[<num>]|.
\end{itemize}

The author names and affiliations could be formatted in two ways:
\begin{enumerate}
\item Group the authors per affiliation.
\item Use an explicit mark to indicate the affiliations.
\end{enumerate}

Author block example:
\begin{lstlisting}
\author[1,2]{Author Name}[%
    prefix=Prof.,
    degree=D.Sc.,
    role=Researcher,
    orcid=0000-0000-000-0000,
    email=name@example.com,
    url=https://name.example.com
]

\address[1]{Affiliation #1}
\address[2]{Affiliation #2}
\end{lstlisting}

\subsection{Abstract and Keywords}

Abstract shall be entered in an environment that starts
with \verb|\begin{abstract}| and ends with
\verb|\end{abstract}|. 

\begin{lstlisting}
\begin{abstract}
  This is an abstract.
\end{abstract}
\end{lstlisting}

The key words are enclosed in a \verb|keywords|
environment. Use \verb|\sep| to separate keywords.

\begin{lstlisting}
\begin{keywords}
  First keyword \sep 
  Second keyword \sep 
  Third keyword \sep 
  Fourth keyword
\end{keywords}
\end{lstlisting}

At the end of front matter add \verb|\maketitle| command.

\subsection{Various Marks in the Front Matter}

The front matter becomes complicated due to various kinds
of notes and marks to the title and author names. Marks in
the title will be denoted by a star ($\star$) mark;
footnotes are denoted by super scripted Arabic numerals,
corresponding author by an Conformal asterisk (*) mark.

\subsubsection{Title marks}

Title mark can be entered by the command, \verb|\tnotemark[<num>]|
and the corresponding text can be entered with the command
\verb|\tnotetext[<num>]{<text>}|. An example will be:

\begin{lstlisting}
\title{A better way to format your document for CEUR-WS}

\tnotemark[1]
\tnotetext[1]{You can use this document as the template for preparing your
  publication. We recommend using the latest version of the ceurart style.}
\end{lstlisting}

\verb|\tnotemark| and \verb|\tnotetext| can be anywhere in
the front matter, but should be before \verb|\maketitle| command.

\subsubsection{Author marks}

Author names can have some kinds of marks and notes:
\begin{itemize}
\item footnote mark : \verb|\fnmark[<num>]|
\item footnote text : \verb|\fntext[<num>]{<text>}|
\item corresponding author mark : \verb|\cormark[<num>]|
\item corresponding author text : \verb|\cortext[<num>]{<text>}|
\end{itemize}

\subsubsection{Other marks}

At times, authors want footnotes which leave no marks in
the author names. The note text shall be listed as part of
the front matter notes. Class files provides
\verb|\nonumnote| for this purpose. The usage
\begin{lstlisting}
\nonumnote{<text>}
\end{lstlisting}
and should be entered anywhere before the \verb|\maketitle|
command for this to take effect. 

\section{Sectioning Commands}

Your work should use standard \LaTeX{} sectioning commands:
\verb|\section|, \verb|\subsection|,
\verb|\subsubsection|, and
\verb|\paragraph|. They should be numbered; do not remove
the numbering from the commands.

Simulating a sectioning command by setting the first word or words of
a paragraph in boldface or italicized text is not allowed.

\section{Tables}

The ``\verb|ceurart|'' document class includes the ``\verb|booktabs|''
package --- \url{https://ctan.org/pkg/booktabs} --- for preparing
high-quality tables.

Table captions are placed \textit{above} the table.

Because tables cannot be split across pages, the best placement for
them is typically the top of the page nearest their initial cite.  To
ensure this proper ``floating'' placement of tables, use the
environment \verb|table| to enclose the table's contents and the
table caption. The contents of the table itself must go in the
\verb|tabular| environment, to be aligned properly in rows and
columns, with the desired horizontal and vertical rules.

Immediately following this sentence is the point at which
Table~\ref{tab:freq} is included in the input file; compare the
placement of the table here with the table in the printed output of
this document.

\begin{table*}
  \caption{Frequency of Special Characters}
  \label{tab:freq}
  \begin{tabular}{ccl}
    \toprule
    Non-English or Math&Frequency&Comments\\
    \midrule
    \O & 1 in 1,000& For Swedish names\\
    $\pi$ & 1 in 5& Common in math\\
    \$ & 4 in 5 & Used in business\\
    $\Psi^2_1$ & 1 in 40,000& Unexplained usage\\
  \bottomrule
\end{tabular}
\end{table*}

To set a wider table, which takes up the whole width of the page's
live area, use the environment \verb|table*| to enclose the table's
contents and the table caption.  As with a single-column table, this
wide table will ``float'' to a location deemed more
desirable. Immediately following this sentence is the point at which
Table~\ref{tab:commands} is included in the input file; again, it is
instructive to compare the placement of the table here with the table
in the printed output of this document.

\begin{table}
  \caption{Some Typical Commands}
  \label{tab:commands}
  \begin{tabular}{ccl}
    \toprule
    Command &A Number & Comments\\
    \midrule
    \texttt{{\char'134}author} & 100& Author \\
    \texttt{{\char'134}table}& 300 & For tables\\
    \texttt{{\char'134}table*}& 400& For wider tables\\
    \bottomrule
  \end{tabular}
\end{table}

\section{Math Equations}

You may want to display math equations in three distinct styles:
inline, numbered or non-numbered display.  Each of the three are
discussed in the next sections.

\subsection{Inline (In-text) Equations}

A formula that appears in the running text is called an inline or
in-text formula.  It is produced by the \verb|math| environment,
which can be invoked with the usual
\verb|\begin| \ldots \verb|\end| construction or with
the short form \verb|$| \ldots \verb|$|. You can use any of the symbols
and structures, from $\alpha$ to $\omega$, available in
\LaTeX~\cite{Lamport:LaTeX};
this section will simply show a few
examples of in-text equations in context. Notice how this equation:
\begin{math}
  \lim_{n\rightarrow \infty} \frac{1}{n} = 0,
\end{math}
set here in in-line math style, looks slightly different when
set in display style.  (See next section).

\subsection{Display Equations}

A numbered display equation---one set off by vertical space from the
text and centered horizontally---is produced by the \verb|equation|
environment. An unnumbered display equation is produced by the
\verb|displaymath| environment.

Again, in either environment, you can use any of the symbols and
structures available in \LaTeX{}; this section will just give a couple
of examples of display equations in context.  First, consider the
equation, shown as an inline equation above:
\begin{equation}
  \lim_{n\rightarrow \infty} \frac{1}{n} = 0.
\end{equation}
Notice how it is formatted somewhat differently in
the \verb|displaymath|
environment.  Now, we'll enter an unnumbered equation:
\begin{displaymath}
  S_{n} = \sum_{i=1}^{n} x_{i} ,
\end{displaymath}
and follow it with another numbered equation:
\begin{equation}
  \lim_{x \to 0} (1 + x)^{1/x} = e
\end{equation}
just to demonstrate \LaTeX's able handling of numbering.

\section{Figures}

The ``\verb|figure|'' environment should be used for figures. One or
more images can be placed within a figure. If your figure contains
third-party material, you must clearly identify it as such, as shown
in the example below.
\begin{figure}
  \centering
  \includegraphics[width=\linewidth-80pt]{UML_state_machine_Fig1}
  \caption{Sample UML state machine (source: \url{https://en.wikipedia.org/wiki/UML_state_machine}, license CC BY-SA 3.0).}
\end{figure}

Your figures should contain a caption which describes the figure to
the reader. Figure captions go below the figure. Your figures should
also include a description suitable for screen readers, to
assist the visually-challenged to better understand your work.

Figure captions are placed below the figure.

\section{Citations and Bibliographies}

The use of Bib\TeX{} for the preparation and formatting of one's
references is strongly recommended. Authors' names should be complete
--- use full first names (``Donald E. Knuth'') not initials
(``D. E. Knuth'') --- and the salient identifying features of a
reference should be included: title, year, volume, number, pages,
article DOI, etc.

The bibliography is included in your source document with these two
commands, placed just before the \verb|\end{document}|
command:
\begin{lstlisting}
\bibliography{bibfile}
\end{lstlisting}
where ``\verb|bibfile|'' is the name, without the ``\verb|.bib|''
suffix, of the Bib\TeX{} file.


\subsection{Some examples}

A paginated journal article \cite{Abril07}, an enumerated journal
article \cite{Cohen07}, a reference to an entire issue
\cite{JCohen96}, a monograph (whole book) \cite{Kosiur01}, a
monograph/whole book in a series (see 2a in spec. document)
\cite{Harel79}, a divisible-book such as an anthology or compilation
\cite{Editor00} followed by the same example, however we only output
the series if the volume number is given \cite{Editor00a} (so series
should not be present since it has no vol. no.), a chapter in a
divisible book \cite{Spector90}, a chapter in a divisible book in a
series \cite{Douglass98}, a multi-volume work as book \cite{Knuth97},
an article in a proceedings (of a conference, symposium, workshop for
example) (paginated proceedings article) \cite{Andler79}, a
proceedings article with all possible elements \cite{Smith10}, an
example of an enumerated proceedings article \cite{VanGundy07}, an
informally published work \cite{Harel78}, a doctoral dissertation
\cite{Clarkson85}, a master's thesis: \cite{anisi03}, an online
document / world wide web resource \cite{Thornburg01, Ablamowicz07,
  Poker06}, a video game (Case 1) \cite{Obama08} and (Case 2)
\cite{Novak03} and \cite{Lee05} and (Case 3) a patent
\cite{JoeScientist001}, work accepted for publication \cite{rous08},
prolific author \cite{SaeediMEJ10} and \cite{SaeediJETC10}. Other
cites might contain `duplicate' DOI and URLs (some SIAM articles)
\cite{Kirschmer:2010:AEI:1958016.1958018}. Multi-volume works as books
\cite{MR781536} and \cite{MR781537}. A couple of citations with DOIs:
\cite{2004:ITE:1009386.1010128,Kirschmer:2010:AEI:1958016.1958018}. Online
citations: \cite{TUGInstmem, Thornburg01, R, UMassCitations}.

\section{Acknowledgments}

Identification of funding sources and other support, and thanks to
individuals and groups that assisted in the research and the
preparation of the work should be included in an acknowledgment
section, which is placed just before the reference section in your
document.

This section has a special environment:
\begin{lstlisting}
\begin{acknowledgments}
  These are different acknowledgments.
\end{acknowledgments}
\end{lstlisting}
so that the information contained therein can be more easily collected
during the article metadata extraction phase, and to ensure
consistency in the spelling of the section heading.

Authors should not prepare this section as a numbered or unnumbered
\verb|\section|; please use the ``\verb|acknowledgments|'' environment.

\section{Appendices}

If your work needs an appendix, add it before the
``\verb|\end{document}|'' command at the conclusion of your source
document.

Start the appendix with the ``\verb|\appendix|'' command:
\begin{lstlisting}
\appendix
\end{lstlisting}
and note that in the appendix, sections are lettered, not
numbered. 

%%
%% The acknowledgments section is defined using the "acknowledgments" environment
%% (and NOT an unnumbered section). This ensures the proper
%% identification of the section in the article metadata, and the
%% consistent spelling of the heading.
\begin{acknowledgments}
  Thanks to the developers of ACM consolidated LaTeX styles
  \url{https://github.com/borisveytsman/acmart} and to the developers
  of Elsevier updated \LaTeX{} templates
  \url{https://www.ctan.org/tex-archive/macros/latex/contrib/els-cas-templates}.  
\end{acknowledgments}

%% The declaration on generative AI comes in effect
%% in Janary 2025. See also
%% https://ceur-ws.org/GenAI/Policy.html
\section*{Declaration on Generative AI}
  {\em Either:}\newline
  The author(s) have not employed any Generative AI tools.
  \newline
  
 \noindent{\em Or (by using the activity taxonomy in ceur-ws.org/genai-tax.html):\newline}
 During the preparation of this work, the author(s) used X-GPT-4 and Gramby in order to: Grammar and spelling check. Further, the author(s) used X-AI-IMG for figures 3 and 4 in order to: Generate images. After using these tool(s)/service(s), the author(s) reviewed and edited the content as needed and take(s) full responsibility for the publication’s content. 

%%
%% Define the bibliography file to be used
\bibliography{sample-ceur}

%%
%% If your work has an appendix, this is the place to put it.
\appendix

\section{Online Resources}


The sources for the ceur-art style are available via
\begin{itemize}
\item \href{https://github.com/yamadharma/ceurart}{GitHub},
% \item \href{https://www.overleaf.com/project/5e76702c4acae70001d3bc87}{Overleaf},
\item
  \href{https://www.overleaf.com/latex/templates/template-for-submissions-to-ceur-workshop-proceedings-ceur-ws-dot-org/pkfscdkgkhcq}{Overleaf
    template}.
\end{itemize}

\end{document}

%%
%% End of file

